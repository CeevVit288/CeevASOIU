%% -*- coding: utf-8 -*-
\documentclass[12pt,a4paper]{scrartcl} 
\usepackage[utf8]{inputenc}
\usepackage[english,russian]{babel}
\usepackage{indentfirst}
\usepackage{misccorr}
\usepackage{graphicx}
\usepackage{amsmath}
\begin{document}
	\begin{titlepage}
		\begin{center}
			\large
			МИНИСТЕРСТВО НАУКИ И ВЫСШЕГО ОБРАЗОВАНИЯ РОССИЙСКОЙ ФЕДЕРАЦИИ
			
			Федеральное государственное бюджетное образовательное учреждение высшего образования
			
			\textbf{АДЫГЕЙСКИЙ ГОСУДАРСТВЕННЫЙ УНИВЕРСИТЕТ}
			\vspace{0.25cm}
			
			Инженерно-физический факультет
			
			Кафедра автоматизированных систем обработки информации и управления
			\vfill

			\vfill
			
			\textsc{Отчет по практике}\\[5mm]
			
			\LARGE\textit{Вариант 5}
			
			{\LARGE Нахождение обратной матрицы}
			\bigskip
			
			1 курс, группа 1ИВТ АСОИУ
		\end{center}
		\vfill
		
		\newlength{\ML}
		\settowidth{\ML}{«\underline{\hspace{0.7cm}}» \underline{\hspace{2cm}}}
		\hfill\begin{minipage}{0.5\textwidth}
			Выполнил:\\
			\underline{\hspace{\ML}} В.\,Т.~Цеев\\
			«\underline{\hspace{0.7cm}}» \underline{\hspace{2cm}} 2024 г.
		\end{minipage}%
		\bigskip
		
		\hfill\begin{minipage}{0.5\textwidth}
			Руководитель:\\
			\underline{\hspace{\ML}} С.\,В.~Теплоухов\\
			«\underline{\hspace{0.7cm}}» \underline{\hspace{2cm}} 2024 г.
		\end{minipage}%
		
		
		\vfill
		
		
		
		\begin{center}
			
			Майкоп, 2024 г.
		\end{center}
	\end{titlepage}
\LARGE{Содержание}

\begin{enumerate}
	\item Задача
	\item Пример кода, решающего данную задачу
	\item Скриншот работы программы
\end{enumerate}
\section{Задача}
Вычислить матрицу обратную заданной.
\section{Пример кода}
\label{sec:exp:code}
\begin{verbatim}
#include <iostream>

void inversion(double** A, int N)
{
	double temp;
	
	double** E = new double* [N];
	
	for (int i = 0; i < N; i++)
	E[i] = new double[N];
	
	for (int i = 0; i < N; i++)
	for (int j = 0; j < N; j++)
	{
		E[i][j] = 0.0;
		
		if (i == j)
		E[i][j] = 1.0;
	}
	
	for (int k = 0; k < N; k++)
	{
		temp = A[k][k];
		
		for (int j = 0; j < N; j++)
		{
			A[k][j] /= temp;
			E[k][j] /= temp;
		}
		
		for (int i = k + 1; i < N; i++)
		{
			temp = A[i][k];
			
			for (int j = 0; j < N; j++)
			{
				A[i][j] -= A[k][j] * temp;
				E[i][j] -= E[k][j] * temp;
			}
		}
	}
	
	for (int k = N - 1; k > 0; k--)
	{
		for (int i = k - 1; i >= 0; i--)
		{
			temp = A[i][k];
			
			for (int j = 0; j < N; j++)
			{
				A[i][j] -= A[k][j] * temp;
				E[i][j] -= E[k][j] * temp;
			}
		}
	}
	
	for (int i = 0; i < N; i++)
	for (int j = 0; j < N; j++)
	A[i][j] = E[i][j];
	
	for (int i = 0; i < N; i++)
	delete[] E[i];
	
	delete[] E;
}

int main()
{
	int N;
	
	std::cout << "Enter N: ";
	std::cin >> N;
	
	double** matrix = new double* [N];
	
	for (int i = 0; i < N; i++)
	matrix[i] = new double[N];
	
	for (int i = 0; i < N; i++)
	for (int j = 0; j < N; j++)
	{
		std::cout << "Enter matrix[" << i << "]
		[" << j << "] = ";
		std::cin >> matrix[i][j];
	}
	
	inversion(matrix, N);
	
	for (int i = 0; i < N; i++)
	{
		for (int j = 0; j < N; j++)
		std::cout << matrix[i][j] << "  ";
		
		std::cout << std::endl;
	}
	
	for (int i = 0; i < N; i++)
	delete[] matrix[i];
	
	delete[] matrix;
	
	std::cin.get();
	return 0;
}
	
	
	
\end{verbatim}
\vfill


\section{Скриншот работы программы}
\label{sec:picexample}
\begin{figure}[h]
	\centering
	\includegraphics[width=0.9\textwidth]{practic.png}
	\caption{Результат}\label{fig:par}
\end{figure}
\end{document}